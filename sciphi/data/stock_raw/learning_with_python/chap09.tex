% LaTeX source for textbook ``How to think like a computer scientist'
% Copyright (c)  2001  Allen B. Downey, Jeffrey Elkner, and Chris Meyers.

% Permission is granted to copy, distribute and/or modify this
% document under the terms of the GNU Free Documentation License,
% Version 1.1  or any later version published by the Free Software
% Foundation; with the Invariant Sections being "Contributor List",
% with no Front-Cover Texts, and with no Back-Cover Texts. A copy of
% the license is included in the section entitled "GNU Free
% Documentation License".

% This distribution includes a file named fdl.tex that contains the text
% of the GNU Free Documentation License.  If it is missing, you can obtain
% it from www.gnu.org or by writing to the Free Software Foundation,
% Inc., 59 Temple Place - Suite 330, Boston, MA 02111-1307, USA.



\chapter{Tuples}
\label{tuplechap}
\index{tuple}

\section{Mutability and tuples}
\index{tuple}
\index{data type!tuple}
\index{type!tuple}
\index{data type!immutable}

So far, you have seen two compound types: strings, which are made up
of characters; and lists, which are made up of elements of any type.
One of the differences we noted is that the elements of a list can be
modified, but the characters in a string cannot.  In other words, strings
are {\bf immutable} and lists are {\bf mutable}.

\index{mutable}
\index{immutable}

There is another type in Python called a {\bf tuple} that is similar
to a list except that it is immutable.  Syntactically, a tuple is a
comma-separated list of values:

\beforeverb
\begin{verbatim}
>>> tuple = 'a', 'b', 'c', 'd', 'e'
\end{verbatim}
\afterverb
%
Although it is not necessary, it is conventional to enclose tuples in
parentheses:

\beforeverb
\begin{verbatim}
>>> tuple = ('a', 'b', 'c', 'd', 'e')
\end{verbatim}
\afterverb
%
To create a tuple with a single element, we have to include the final
comma:

\beforeverb
\begin{verbatim}
>>> t1 = ('a',)
>>> type(t1)
<type 'tuple'>
\end{verbatim}
\afterverb
%
Without the comma, Python treats {\tt ('a')} as a string in
parentheses:

\beforeverb
\begin{verbatim}
>>> t2 = ('a')
>>> type(t2)
<type 'str'>
\end{verbatim}
\afterverb
%
Syntax issues aside, the operations on tuples are the same as the
operations on lists.  The index operator selects an element from
a tuple.

\beforeverb
\begin{verbatim}
>>> tuple = ('a', 'b', 'c', 'd', 'e')
>>> tuple[0]
'a'
\end{verbatim}
\afterverb
%
And the slice operator selects a range of elements.

\beforeverb
\begin{verbatim}
>>> tuple[1:3]
('b', 'c')
\end{verbatim}
\afterverb
%
But if we try to modify one of the elements of the tuple, we get
an error:

\index{runtime error}

\beforeverb
\begin{verbatim}
>>> tuple[0] = 'A'
TypeError: object doesn't support item assignment
\end{verbatim}
\afterverb
%
Of course, even if we can't modify the elements of a tuple, we can
replace it with a different tuple:

\beforeverb
\begin{verbatim}
>>> tuple = ('A',) + tuple[1:]
>>> tuple
('A', 'b', 'c', 'd', 'e')
\end{verbatim}
\afterverb
%

\section{Tuple assignment}
\label{tuple assignment}
\index{tuple assignment}
\index{assignment!tuple}

Once in a while, it is useful to swap the values of two variables.
With conventional assignment statements, we have to use a temporary
variable.  For example, to swap {\tt a} and {\tt b}:

\beforeverb
\begin{verbatim}
>>> temp = a
>>> a = b
>>> b = temp
\end{verbatim}
\afterverb
%
If we have to do this often, this approach becomes cumbersome.  Python
provides a form of {\bf tuple assignment} that solves this problem neatly:

\beforeverb
\begin{verbatim}
>>> a, b = b, a
\end{verbatim}
\afterverb
%
The left side is a tuple of variables; the right side is a tuple of
values.  Each value is assigned to its respective variable.  
All the expressions on the right side are evaluated before any
of the assignments.
This feature makes tuple
assignment quite versatile.

Naturally, the number of variables on the left and the number of
values on the right have to be the same:

\beforeverb
\begin{verbatim}
>>> a, b, c, d = 1, 2, 3
ValueError: unpack tuple of wrong size
\end{verbatim}
\afterverb
%

\section{Tuples as return values}
\index{tuple}
\index{value!tuple}
\index{return value!tuple}
\index{function!tuple as return value}

Functions can return tuples as return values.  For example, we could
write a function that swaps two parameters:

\beforeverb
\begin{verbatim}
def swap(x, y):
  return y, x
\end{verbatim}
\afterverb
%
Then we can assign the return value to a
tuple with two variables:

\beforeverb
\begin{verbatim}
a, b = swap(a, b)
\end{verbatim}
\afterverb
%
In this case, there is no great advantage in making {\tt swap} a
function.  In fact, there is a danger in trying to encapsulate {\tt
swap}, which is the following tempting mistake:

\beforeverb
\begin{verbatim}
def swap(x, y):      # incorrect version
  x, y = y, x
\end{verbatim}
\afterverb
%
If we call this function like this:

\beforeverb
\begin{verbatim}
swap(a, b)
\end{verbatim}
\afterverb
%
then {\tt a} and {\tt x} are aliases for the same value.  Changing {\tt x}
inside {\tt swap} makes {\tt x} refer to a different value, but it has no
effect on {\tt a} in {\tt \_\_main\_\_}.  Similarly, changing {\tt y} has no
effect on {\tt b}.

This function runs without producing an error message, but it
doesn't do what we intended.  This is an example of a semantic
error.

\index{semantic error}

\begin{quote}
{\em As an exercise, draw a state diagram for this function so that
you can see why it doesn't work.}
\end{quote}


\section{Random numbers}
\index{random number}
\index{number!random}

Most computer programs do the same thing every time they execute,
so they are said to be {\bf deterministic}.  Determinism is usually a
good thing, since we expect the same calculation to yield the same
result.  For some applications, though, we want the computer to
be unpredictable.  Games are an obvious example, but there are
more.

Making a program truly nondeterministic turns out to be not so easy,
but there are ways to make it at least seem nondeterministic.  One of
them is to generate random numbers and use them to determine the
outcome of the program.  Python provides a built-in function that
generates {\bf pseudorandom} numbers, which are not truly random in
the mathematical sense, but for our purposes they will do.

The {\tt random} module contains a function called {\tt random} that
returns a floating-point number between 0.0 and 1.0.  Each time you
call {\tt random}, you get the next number in a long series.  To see a
sample, run this loop:

\beforeverb
\begin{verbatim}
import random

for i in range(10):
  x = random.random()
  print x
\end{verbatim}
\afterverb
%
To generate a random number between 0.0 and an upper bound like
{\tt high}, multiply {\tt x} by {\tt high}.

\begin{quote}
{\em As an exercise, generate a random number between {\tt low} and
{\tt high}.}
\end{quote}

\begin{quote}
{\em As an additional exercise, generate a random {\em integer}
between {\tt low} and {\tt high}, including both end points.}
\end{quote}


\section{List of random numbers}

The first step is to generate a list of random values.  {\tt
randomList} takes an integer argument and returns a list of random
numbers with the given length.  It starts with a list of {\tt n}
zeros.  Each time through the loop, it replaces one of the elements
with a random number.
The return value is a reference to the complete list:

\beforeverb
\begin{verbatim}
def randomList(n):
  s = [0] * n
  for i in range(n):
    s[i] = random.random()
  return s
\end{verbatim}
\afterverb
%
We'll test this function with a list of eight elements.  For
purposes of debugging, it is a good idea to start small.

\adjustpage{-2}
\pagebreak

\beforeverb
\begin{verbatim}
>>> randomList(8)
0.15156642489
0.498048560109
0.810894847068
0.360371157682
0.275119183077
0.328578797631
0.759199803101
0.800367163582
\end{verbatim}
\afterverb
%
The numbers generated by {\tt random} are supposed to be distributed
uniformly, which means that every value is equally likely.

If we divide the range of possible
values into equal-sized ``buckets,'' and count the number of times a
random value falls in each bucket, we should get roughly the
same number in each.

We can test this theory by writing a program to 
divide the range into
buckets and count the number of values in each.


\section{Counting}
\index{counting}

A good approach to problems like this is to divide the problem into
subproblems and look for subproblems that fit a computational pattern
you have seen before.

In this case, we want to traverse a list of numbers and count the
number of times a value falls in a given range.  That sounds familiar.
In Section~\ref{counter}, we wrote a program that traversed a string and
counted the number of times a given letter appeared.

So, we can proceed by copying the old program and adapting it
for the current problem.  The original program was:

\beforeverb
\begin{verbatim}
count = 0
for char in fruit:
  if char == 'a':
    count = count + 1
print count
\end{verbatim}
\afterverb
%
The first step is to replace {\tt fruit} with {\tt t} and
{\tt char} with {\tt num}.  That doesn't change the program;
it just makes it more readable.

The second step is to change the test.  We aren't interested
in finding letters.  We want to see if {\tt num} is between
the given values {\tt low} and {\tt high}.  

\beforeverb
\begin{verbatim}
count = 0
for num in t:
  if low < num < high:
    count = count + 1
print count
\end{verbatim}
\afterverb
%
The last step is to encapsulate this code in a function called
{\tt inBucket}.  The parameters are the list and the values
{\tt low} and {\tt high}.

\beforeverb
\begin{verbatim}
def inBucket(t, low, high):
  count = 0
  for num in t:
    if low < num < high:
      count = count + 1
  return count
\end{verbatim}
\afterverb
%
By copying and modifying an existing program, we were able
to write this function quickly and save a lot of debugging
time.  This development plan is called {\bf pattern matching}.
If you find yourself working on a problem you have solved
before, reuse the solution.


\section{Many buckets}

As the number of buckets increases, {\tt inBucket} gets
a little unwieldy.  With two buckets, it's not bad:

\beforeverb
\begin{verbatim}
low = inBucket(a, 0.0, 0.5)
high = inBucket(a, 0.5, 1)
\end{verbatim}
\afterverb
%
But with four buckets it is getting cumbersome.

\beforeverb
\begin{verbatim}
bucket1 = inBucket(a, 0.0, 0.25)
bucket2 = inBucket(a, 0.25, 0.5)
bucket3 = inBucket(a, 0.5, 0.75)
bucket4 = inBucket(a, 0.75, 1.0)
\end{verbatim}
\afterverb
%
There are two problems.  One is that we have to make up new
variable names for each result.  The other is that we have to
compute the range for each bucket.

We'll solve the second problem first.  If the number of buckets
is {\tt numBuckets}, then the width of each bucket is
{\tt 1.0 / numBuckets}.

We'll use a loop to compute the range of each bucket.
The loop variable, {\tt i},
counts from 0 to {\tt numBuckets-1}:

\beforeverb
\begin{verbatim}
bucketWidth = 1.0 / numBuckets
for i in range(numBuckets):
  low = i * bucketWidth
  high = low + bucketWidth
  print low, "to", high
\end{verbatim}
\afterverb
%
To compute the low end of each bucket, we multiply the loop variable
by the bucket width.  The high end is just a {\tt bucketWidth} away.

With {\tt numBuckets = 8}, the output is:

\beforeverb
\begin{verbatim}
0.0 to 0.125
0.125 to 0.25
0.25 to 0.375
0.375 to 0.5
0.5 to 0.625
0.625 to 0.75
0.75 to 0.875
0.875 to 1.0
\end{verbatim}
\afterverb
%
You can confirm that each bucket is the same width, that they don't
overlap, and that they cover the entire range from 0.0 to 1.0.

Now back to the first problem.
We need a way to store eight integers, using the loop variable
to indicate one at a time.  By now you should be thinking,
``List!''

We have to create the bucket list outside the loop, because we only
want to do it once.  Inside the loop, we'll call {\tt inBucket}
repeatedly and update the {\tt i}-eth element of the list:

\beforeverb
\begin{verbatim}
numBuckets = 8
buckets = [0] * numBuckets
bucketWidth = 1.0 / numBuckets
for i in range(numBuckets):
  low = i * bucketWidth
  high = low + bucketWidth
  buckets[i] = inBucket(t, low, high)
print buckets
\end{verbatim}
\afterverb
%
With a list of 1000 values, this code produces this bucket list:

\beforeverb
\begin{verbatim}
[138, 124, 128, 118, 130, 117, 114, 131]
\end{verbatim}
\afterverb
%
These numbers are fairly close to 125, which is what we expected.  At
least, they are close enough that we can believe the random number
generator is working.

\begin{quote}
{\em As an exercise, 
test this function with some longer lists, and see if the
number of values in each bucket tends to level off.}
\end{quote}


\section{A single-pass solution}
\index{histogram}

Although this program works, it is not as efficient as it could be.
Every time it calls {\tt inBucket}, it traverses the entire list.  As
the number of buckets increases, that gets to be a lot of traversals.

It would be better to make a single pass through the list and compute
for each value the index of the bucket in which it falls.  Then we can
increment the appropriate counter.

In the previous section we took an index, {\tt i}, and multiplied it
by the {\tt bucketWidth} to find the lower bound of a given
bucket.  Now we want to take a value in the range 0.0 to 1.0 and find the
index of the bucket where it falls.

Since this problem is the inverse of the previous problem, we might
guess that we should divide by {\tt bucketWidth} instead of
multiplying.  That guess is correct.

Since {\tt bucketWidth = 1.0 / numBuckets}, dividing by {\tt
bucketWidth} is the same as multiplying by {\tt numBuckets}.  If we
multiply a number in the range 0.0 to 1.0 by {\tt numBuckets}, we get
a number in the range from 0.0 to {\tt numBuckets}.  If we round that
number to the next lower integer, we get exactly what we are looking
for---a bucket index:

\beforeverb
\begin{verbatim}
numBuckets = 8
buckets = [0] * numBuckets
for i in t:
  index = int(i * numBuckets)
  buckets[index] = buckets[index] + 1
\end{verbatim}
\afterverb
%
We used the {\tt int} function to convert a floating-point
number to an integer.

Is it possible for this calculation to produce an index that is out of
range (either negative or greater than {\tt len(buckets)-1})?

A list like {\tt buckets} that contains counts of the number of values
in each range is called a {\bf histogram}.

\begin{quote}
{\em As an exercise, write a function called {\tt histogram} that
takes a list and a number of buckets as arguments and returns
a histogram with the given number of buckets.}
\end{quote}

\adjustpage{-2}
\pagebreak

\section{Glossary}

\begin{description}

\item[immutable type:] A type in which the elements cannot be
modified.  Assignments
to elements or slices of immutable types cause an error.

\item[mutable type:] A data type in which the elements can be
modified.  All mutable
types are compound types.  Lists and dictionaries are mutable data
types; strings and tuples are not.

\item[tuple:] A sequence type that is similar to a list except that it is
immutable.  Tuples can be used wherever an immutable type is required, such
as a key in a dictionary.

\item[tuple assignment:] An assignment to all of the elements in a tuple using
a single assignment statement. Tuple assignment occurs in parallel rather
than in sequence, making it useful for swapping values.

\item[deterministic:] A program that does the same thing each time it is
called.

\item[pseudorandom:] A sequence of numbers that appear to be random but that
are actually the result of a deterministic computation.

\item[histogram:] A list of integers in which each element counts the 
number of times something happens.

\item[pattern matching:] A program development plan that involves
identifying a familiar computational pattern and copying the
solution to a similar problem.

\index{mutable type}
\index{immutable type}
\index{tuple}
\index{tuple assignment}
\index{assignment!tuple}
\index{deterministic}
\index{pseudorandom}
\index{histogram}
\index{pattern matching}

\end{description}

